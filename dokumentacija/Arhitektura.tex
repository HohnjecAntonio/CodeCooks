\chapter{Arhitektura i dizajn sustava}
		
		\textbf{\textit{dio 1. revizije}}\\

		\textit{ Potrebno je opisati stil arhitekture te identificirati: podsustave, preslikavanje na radnu platformu, spremišta podataka, mrežne protokole, globalni upravljački tok i sklopovsko-programske zahtjeve. Po točkama razraditi i popratiti odgovarajućim skicama:}
	\begin{itemize}
		\item 	\textit{izbor arhitekture temeljem principa oblikovanja pokazanih na predavanjima (objasniti zašto ste baš odabrali takvu arhitekturu)}
		\item 	\textit{organizaciju sustava s najviše razine apstrakcije (npr. klijent-poslužitelj, baza podataka, datotečni sustav, grafičko sučelje)}
		\item 	\textit{organizaciju aplikacije (npr. slojevi frontend i backend, MVC arhitektura) }		
	\end{itemize}


		\noindent U nastavku opisujemo arhitekturu i organizaciju sustava na apstraktnoj razini te organizaciju aplikacije na razini slojeva backend-a i frontend-a koristeči MVC arhitekturu.


		\eject
		

		

				
		\section{Baza podataka}
		
		
		\textnormal{Kao rješenje za web aplikaciju koju izrađujemo, odlučili smo se na relacijski model baze podataka jer nam ona omogućava precizno oblikovanje i modeliranje elemenata iz stvarnog svijeta. Međusobni odnos tablica se zasniva na relaciji dok se svaka tablica sastoji od naziva entiteta te njegovih pripadajućih atributa koji opisuju dani entitet. Ovakav tip baze podataka u ovom slučaju nam omogućava brzo spremanje, dohvat i izmjenu (uređivanje ili brisanje) podataka koji cirkulišu u web aplikaciji. Baza podataka koja je spremna za izradu ove web aplikacije se sastoji od sljedećih elemenata/tablica:}
		
		\begin{packed_item}
			
			\item Korisnik
			\item Recept
			\item Video
			\item Objava
			\item Kategorija
			\item ReceptKategorije
			\item ReceptSastojci
			\item Sastojak
			\item Komentar
			\item OmiljeniAutor
			\item OznačenRecept
			\item SpremljenRecept
			\item VrstaKuhinja
			\item Obavijesti
			
		\end{packed_item}
		
		\eject
		
		\subsection{Opis tablica}
		
		
		\textnormal{\textbf{Korisnik}		Entitet Korisnik služi za evidenciju podataka o korisnicima koji koriste web aplikaciju. Atributi koji ga opisuju su korisnickoIme, lozinkaKorisnik, imeKorisnik, prezimeKorisnik, brojTelefona, emailKorisnik, razinaOvlasti i Dostupan što označava kada je dostupan za komunikaciju s drugim korisnicima unutar web aplikacije. Entitet Korisnik je u \textit{One-To-Many} vezi s entitetom Recept. Entitet Korisnik je i u \textit{Many-To-Many} vezi s entitetom OmiljeniAutor koji bilježi korisnike koje ovaj korisnik prati. Također, ovaj entitet je u \textit{Many-To-Many} vezi s entitetima SpremljenRecept, OznačenRecept, a u \textit{One-To-Many} vezi s entitetima Objava i Komentar.}
		
		
		\begin{longtblr}[
			label=none,
			entry=none
			]{
				width = \textwidth,
				colspec={|X[8,l]|X[6, l]|X[20, l]|},
				rowhead = 1,
			} %definicija širine tablice, širine stupaca, poravnanje i broja redaka naslova tablice
			\hline \SetCell[c=3]{c}{\textbf{Korisnik}}                                     \\ \hline[3pt]
			\SetCell{LightGreen}IDKorisnik & INT     & jedinstveni identifikator korisnika \\ \hline
			KorisnickoIme                  & VARCHAR & korisničko ime                      \\ \hline
			LozinkaKorisnik                & VARCHAR & spremljena hash lozinka za pristup  \\ \hline
			ImeKorisnik                    & VARCHAR & ime korisnika                       \\ \hline
			PrezimeKorisnik                & VARCHAR & prezime korisnika                   \\ \hline
			BrojTelefona                   & VARCHAR & broj telefona korisnika             \\ \hline
			EmailKorisnik                  & VARCHAR & e-mail adresa korisnika             \\ \hline
			RazinaOvlasti                  & VARCHAR & razina ovlasti u aplikaciji         \\ \hline
			Dostupan                       & TIME    & vrijeme dostupnosti                 \\ \hline
		\end{longtblr}
		
		\eject
		
		\textnormal{\textbf{Komentar}		Entitet Komentar služi za evidenciju komentara koji korisnici međusobno dijele u web aplikaciji. Opisuju ga atributi IDKomentar, IDKorisnik, IDObjava, te NaslovKomentar, SadrzajKomentar i DatumKomentar. Ovaj entitet je  u \textit{Many-To-One} vezi s entitetom Korisnik, te u istoj vezi s entitetom Objava.}
		
		\begin{longtblr}[
			label=none,
			entry=none
			]{
				width = \textwidth,
				colspec={|X[8,l]|X[6, l]|X[20, l]|},
				rowhead = 1,
			} %definicija širine tablice, širine stupaca, poravnanje i broja redaka naslova tablice
			\hline \SetCell[c=3]{c}{\textbf{Komentar}}                                     \\ \hline[3pt]
			\SetCell{LightGreen}IDKomentar & INT     & jedinstveni identifikator komentara \\ \hline
			IDKorisnik                     & INT     & jedinstveni identifikator korisnika \\ \hline
			IDObjava                       & INT     & jedinstveni identifikator objave    \\ \hline
			NaslovKomentar                 & VARCHAR & naziv komentara                     \\ \hline
			OpisKomentar                   & VARCHAR & sadržaj komentara                   \\ \hline
			DatumKomentar                  & DATE    & datum komentara                     \\ \hline
		\end{longtblr}
		
		
		\vspace{\baselineskip}
		\textnormal{\textbf{OmiljeniAutor}		Entitet OmiljeniAutor služi za evidenciju korisnika koje jedan korisnik zaprati. Opisan je sljedećim atributima: IDKorisnik i IDAutor. U \textit{Many-To-Many} je vezi s entitetom Korisnik unutar web aplikacije.}
		
		\begin{longtblr}[
			label=none,
			entry=none
			]{
				width = \textwidth,
				colspec={|X[6,l]|X[6, l]|X[20, l]|},
				rowhead = 1,
			} %definicija širine tablice, širine stupaca, poravnanje i broja redaka naslova tablice
			\hline \SetCell[c=3]{c}{\textbf{OmiljeniAutor}}                           \\ \hline[3pt]
			\SetCell{LightGreen}IDKorisnik & INT & jedinstveni identifikator pratitelja \\ \hline
			IDAutor                        & INT & ID korisnika koji je zapraćen        \\ \hline
		\end{longtblr}
		
		\vspace{\baselineskip}
		\textnormal{\textbf{Recept}		Entitet Recept služi za evidenciju podataka o receptima koji cirkulišu i nastaju u web aplikaciji. Opisan je atributima: IDRecept, NazivRecept, IDKategorija, IDSastojak, IDVrstaKuhinja, PripremaRecept, VrijemeKuhanja, IDOznaka, SlikaRecept i IDVideoRecept. Entitet Recept je u \textit{Many-To-One} vezi s entitetom Korisnik, a u \textit{Many-To-Many} vezi s entitetom Sastojak, te u \textit{Many-To-Many} vezi s Kategorija, a \textit{One-To-One} s VrstaKuhinja. Također je u \textit{One-To-One} vezi s entitetom Objava.}
		
		\begin{longtblr}[
			label=none,
			entry=none
			]{
				width = \textwidth,
				colspec={|X[8,l]|X[6, l]|X[20, l]|},
				rowhead = 1,
			} %definicija širine tablice, širine stupaca, poravnanje i broja redaka naslova tablice
			\hline \SetCell[c=3]{c}{\textbf{Recept}}                                      \\ \hline[3pt]
			\SetCell{LightGreen}IDRecept & INT     & jedinstveni identifikator recepta    \\ \hline
			NazivRecept                  & VARCHAR & naziv recepta                        \\ \hline
			IDKategorija                 & INT     & ID kategorije kojoj recept pripada   \\ \hline
			IDSastojak                   & INT     & ID sastojka u receptu                \\ \hline
			IDVrstaKuhinja               & INT     & ID vrste kuhinje recepta             \\ \hline
			PripremaRecept               & VARCHAR & Opis pripreme recepta                \\ \hline
			VrijemeKuhanja               & TIME    & Vrijeme potrebno za pripremu recepta \\ \hline
			IDOznaka                     & INT     & ID oznake recepta                    \\ \hline
			SlikaRecept                  & VARCHAR & Slika recepta                        \\ \hline
			IDVideoRecept                & INT     & ID Video recepta                     \\ \hline
		\end{longtblr}
		
		\vspace{\baselineskip}
		\textnormal{\textbf{Video}		Entitet Video služi za evidenciju videa koji se objavljuju uz recepte. Opisan je sljedećim atributima: IDVideo, NazivVideo i TrajanjeVideo što omogućuje pretragu recepata po dužini pripremanja. U \textit{One-To-One} je vezi s entitetom Recept unutar web aplikacije.}
		
		\begin{longtblr}[
			label=none,
			entry=none
			]{
				width = \textwidth,
				colspec={|X[6,l]|X[6, l]|X[20, l]|},
				rowhead = 1,
			} %definicija širine tablice, širine stupaca, poravnanje i broja redaka naslova tablice
			\hline \SetCell[c=3]{c}{\textbf{Video}}                       \\ \hline[3pt]
			\SetCell{LightGreen}IDVideo & INT     & jedinstveni identifikator videa \\ \hline
			NazivVideo                  & VARCHAR & naziv videa                     \\ \hline
			TrajanjeVideo               & TIME    & trajanje videa                  \\ \hline
		\end{longtblr}
		
		\vspace{\baselineskip}
		\textnormal{\textbf{Objava}		Entitet Objava služi za evidenciju objava recepata koje korisnici objavljuju unutar web aplikacije. Opisan je atributima: IDObjava, IDKorisnik, IDRecept i DatumObjava. U \textit{Many-To-One} vezi je s entitetom Korisnik, te u \textit{Many-To-Many} vezi s entitetom Komentar i u \textit{One-To-One} s entitetom Recept.}
		
		\begin{longtblr}[
			label=none,
			entry=none
			]{
				width = \textwidth,
				colspec={|X[6,l]|X[6, l]|X[20, l]|},
				rowhead = 1,
			} %definicija širine tablice, širine stupaca, poravnanje i broja redaka naslova tablice
			\hline \SetCell[c=3]{c}{\textbf{Objava}}                                  \\ \hline[3pt]
			\SetCell{LightGreen}IDObjava & INT  & jedinstveni identifikator objave    \\ \hline
			IDKorisnik                   & INT  & jedinstveni identifikator korisnika \\ \hline
			IDRecept                     & INT  & jedinstveni identifikator recepta   \\ \hline
			DatumObjava                  & DATE & datum objave                        \\ \hline
		\end{longtblr}
		
		\vspace{\baselineskip}
		\textnormal{\textbf{OznačenRecept}		Entitet OznačenRecept služi za evidenciju označenih recepata od strane korisnika. Opisan je atributima: IDRecept i IDKorisnik. U \textit{Many-To-Many} vezi je s entitetom Korisnik.}
		
		\begin{longtblr}[
			label=none,
			entry=none
			]{
				width = \textwidth,
				colspec={|X[6,l]|X[6, l]|X[20, l]|},
				rowhead = 1,
			} %definicija širine tablice, širine stupaca, poravnanje i broja redaka naslova tablice
			\hline \SetCell[c=3]{c}{\textbf{OznačenRecept}}                          \\ \hline[3pt]
			\SetCell{LightGreen}IDRecept & INT & jedinstveni identifikator recepta   \\ \hline
			IDKorisnik                   & INT & ID korisnika koji je označio recept \\ \hline
		\end{longtblr}
		
		\vspace{\baselineskip}
		\textnormal{\textbf{SpremljenRecept}		Entitet SpremljenRecept za evidenciju spremljenih recepata od strane korisnika. Opisan je atributima: IDRecept i IDKorisnik. U \textit{Many-To-Many} vezi je s entitetom Korisnik.}
		
		\begin{longtblr}[
			label=none,
			entry=none
			]{
				width = \textwidth,
				colspec={|X[6,l]|X[6, l]|X[20, l]|},
				rowhead = 1,
			} %definicija širine tablice, širine stupaca, poravnanje i broja redaka naslova tablice
			\hline \SetCell[c=3]{c}{\textbf{SpremljenRecept}}                        \\ \hline[3pt]
			\SetCell{LightGreen}IDRecept & INT & jedinstveni identifikator recepta   \\ \hline
			IDKorisnik                   & INT & ID korisnika koji je spremio recept \\ \hline
		\end{longtblr}
		
		\vspace{\baselineskip}
		\textnormal{\textbf{Kategorija}		Entitet Kategorija služi za evidenciju kategorija. Opisan je atributima: IDKategorija, NazivKategorija i OpisKategorija.}
		
		\begin{longtblr}[
			label=none,
			entry=none
			]{
				width = \textwidth,
				colspec={|X[8,l]|X[6, l]|X[20, l]|},
				rowhead = 1,
			} %definicija širine tablice, širine stupaca, poravnanje i broja redaka naslova tablice
			\hline \SetCell[c=3]{c}{\textbf{Kategorija}}                                      \\ \hline[3pt]
			\SetCell{LightGreen}IDKategorija & INT     & jedinstveni identifikator kateogrije \\ \hline
			NazivKategorija                  & VARCHAR & naziv kategorije                     \\ \hline
			OpisKategorija                   & VARCHAR & opis kategorije                      \\ \hline
		\end{longtblr}
		
		\vspace{\baselineskip}
		\textnormal{\textbf{ReceptKategorije}		Entitet ReceptKategorije služi za evidenciju kategorija i recepata pod tim kategorijama. Opisan je atributima: IDKategorija, NazivKategorija i OpisKategorija. U \textit{Many-To-Many} vezi je s entitetom Recept.}
		
		\begin{longtblr}[
			label=none,
			entry=none
			]{
				width = \textwidth,
				colspec={|X[8,l]|X[6, l]|X[20, l]|},
				rowhead = 1,
			} %definicija širine tablice, širine stupaca, poravnanje i broja redaka naslova tablice
			\hline \SetCell[c=3]{c}{\textbf{ReceptKategorije}}                        \\ \hline[3pt]
			\SetCell{LightGreen}IDRecept & INT & jedinstveni identifikator kateogrije \\ \hline
			IDKategorija                 & INT & jedinstveni identifikator recepta    \\ \hline
		\end{longtblr}
		
		\vspace{\baselineskip}
		\textnormal{\textbf{VrstaKuhinja}		Entitet VrstaKuhinja služi za evidenciju različitih vrsta kuhinja. Opisan je atributima: IDVrstaKuhinja, NazivVrstaKuhinja i OpisKategorija. U \textit{One-To-One} vezi je s entitetom Recept.}
		
		\begin{longtblr}[
			label=none,
			entry=none
			]{
				width = \textwidth,
				colspec={|X[9,l]|X[6, l]|X[20, l]|},
				rowhead = 1,
			} %definicija širine tablice, širine stupaca, poravnanje i broja redaka naslova tablice
			\hline \SetCell[c=3]{c}{\textbf{VrstaKuhinja}}                                         \\ \hline[3pt]
			\SetCell{LightGreen}IDVrstaKuhinje & INT     & jedinstveni identifikator vrste kuhinje \\ \hline
			NazivVrstaKuhinja                  & VARCHAR & naziv vrste kuhinje                     \\ \hline
		\end{longtblr}
		
		\vspace{\baselineskip}
		\textnormal{\textbf{Sastojak}		Entitet Sastojak služi za evidenciju različitih sastojaka. Opisan je atributima: IDSastojak, NazivSastojak.}
		
		\begin{longtblr}[
			label=none,
			entry=none
			]{
				width = \textwidth,
				colspec={|X[6,l]|X[6, l]|X[20, l]|},
				rowhead = 1,
			} %definicija širine tablice, širine stupaca, poravnanje i broja redaka naslova tablice
			\hline \SetCell[c=3]{c}{\textbf{Sastojak}}                                    \\ \hline[3pt]
			\SetCell{LightGreen}IDSastojak & INT     & jedinstveni identifikator sastojka \\ \hline
			NazivSastojak                  & VARCHAR & naziv sastojka                     \\ \hline
		\end{longtblr}
		
		\vspace{\baselineskip}
		\textnormal{\textbf{ReceptSastojci}		Entitet ReceptSastojci služi za evidenciju različitih sastojaka unutar recepta. Opisan je atributima: IDRecept, IDSastojak i KolicinaSastojak. U \textit{Many-To-Many} vezi je s entitetom Recept.}
		
		\begin{longtblr}[
			label=none,
			entry=none
			]{
				width = \textwidth,
				colspec={|X[8,l]|X[6, l]|X[20, l]|},
				rowhead = 1,
			} %definicija širine tablice, širine stupaca, poravnanje i broja redaka naslova tablice
			\hline \SetCell[c=3]{c}{\textbf{ReceptSastojci}}                            \\ \hline[3pt]
			\SetCell{LightGreen}IDRecept & INT     & jedinstveni identifikator recepta  \\ \hline
			IDSastojak                   & INT     & jedinstveni identifikator sastojka \\ \hline
			KolicinaSastojak             & VARCHAR & količina sastojka                  \\ \hline
		\end{longtblr}
		
		\vspace{\baselineskip}
		\textnormal{\textbf{Obavijesti}		Entitet Obavijesti služi za slanje obavijesti korisnicima koji prate određene autore recepata. Opisan je atributima: IDObavijest, IDAutor, NazivObavijest, SadrzajObavijest, DatumObavijest i JeProcitano. U \textit{Many-To-One} vezi je s entitetom Korisnik.}
		
		\begin{longtblr}[
			label=none,
			entry=none
			]{
				width = \textwidth,
				colspec={|X[8,l]|X[6, l]|X[20, l]|},
				rowhead = 1,
			} %definicija širine tablice, širine stupaca, poravnanje i broja redaka naslova tablice
			\hline \SetCell[c=3]{c}{\textbf{Obavijesti}}                                    \\ \hline[3pt]
			\SetCell{LightGreen}IDObavijest & INT     & jed. identifikator obavijesti       \\ \hline
			IDKorisnik                      & INT     & jedinstveni identifikator korisnika \\ \hline
			NazivObavijest                  & VARCHAR & naziv obavijesti                    \\ \hline
			SadrzajObavijest                & VARCHAR & sadržaj obavijesti                  \\ \hline
			DatumObavijest                  & DATE    & datum obavijesti                    \\ \hline
			JeProcitano                     & TINYINT & oznaka pročitane obavijesti         \\ \hline
		\end{longtblr}
		
		\subsection{Dijagram baze podataka}
		%unos slike
		\begin{figure}[H]
			\includegraphics[width=\textwidth]{slike/CookBooked-dbp.png} %veličina slike u odnosu na originalnu datoteku i pozicija slike
			\centering
			\caption{Dijagram relacijske baze podataka za web aplikaciju}
			\label{fig:dijagrambp}
		\end{figure}
		\eject
			
			
		\section{Dijagram razreda}
		
			\textit{Potrebno je priložiti dijagram razreda s pripadajućim opisom. Zbog preglednosti je moguće dijagram razlomiti na više njih, ali moraju biti grupirani prema sličnim razinama apstrakcije i srodnim funkcionalnostima.}\\
			
			\textbf{\textit{dio 1. revizije}}\\
			
			\textit{Prilikom prve predaje projekta, potrebno je priložiti potpuno razrađen dijagram razreda vezan uz \textbf{generičku funkcionalnost} sustava. Ostale funkcionalnosti trebaju biti idejno razrađene u dijagramu sa sljedećim komponentama: nazivi razreda, nazivi metoda i vrste pristupa metodama (npr. javni, zaštićeni), nazivi atributa razreda, veze i odnosi između razreda.}\\
			
			\textbf{\textit{dio 2. revizije}}\\			
			
			\textit{Prilikom druge predaje projekta dijagram razreda i opisi moraju odgovarati stvarnom stanju implementacije}
			
			
			
			\eject
		
		\section{Dijagram stanja}
			
			
			\textbf{\textit{dio 2. revizije}}\\
			
			\textit{Potrebno je priložiti dijagram stanja i opisati ga. Dovoljan je jedan dijagram stanja koji prikazuje \textbf{značajan dio funkcionalnosti} sustava. Na primjer, stanja korisničkog sučelja i tijek korištenja neke ključne funkcionalnosti jesu značajan dio sustava, a registracija i prijava nisu. }
			
			
			\eject 
		
		\section{Dijagram aktivnosti}
			
			\textbf{\textit{dio 2. revizije}}\\
			
			 \textit{Potrebno je priložiti dijagram aktivnosti s pripadajućim opisom. Dijagram aktivnosti treba prikazivati značajan dio sustava.}
			
			\eject
		\section{Dijagram komponenti}
		
			\textbf{\textit{dio 2. revizije}}\\
		
			 \textit{Potrebno je priložiti dijagram komponenti s pripadajućim opisom. Dijagram komponenti treba prikazivati strukturu cijele aplikacije.}