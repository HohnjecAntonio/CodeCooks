\chapter{Arhitektura i dizajn sustava}

\textnormal{Arhitektura sustava je bazirana na tri komponente koje komuniciraju jedna s drugom. Odnosno, sustav je podijeljen u tri dolje navedena sloja, a koji se mogu prikazati slikom ispod.}
\begin{itemize}
	\item 	\textit{Web poslužitelj}
	\item 	\textit{Web aplikacija}
	\item 	\textit{Baza podataka}
\end{itemize}

\begin{figure}[H]
	\includegraphics[width=\textwidth]{slike/arhitekturaSustava.png} %veličina slike u odnosu na originalnu datoteku i pozicija slike
	\centering
	\caption{Arhitektura sustava}
	\label{fig:arhitekturasustava}
\end{figure}

\textnormal{\textbf{Web preglednik} omogućava korisniku prikaz web stranice koja pruža određene funkcionalnosti. Omogućava prikaz web stranice (prikaz videa, slika i ostalog multimedijalnog sadržaja) onako kako je definirana u datotekama, dakle omogućuje interpretiranje koda u koristan oblik "običnom" korisniku.
	Putem web preglednika korisnik šalje zahtjev za željenu radnju koja se onda proslijedi idućim koponentama na obradu, te nakon obrade opet se prikazuju u vidljivom obliku kao vid povratne informacije.}

\textnormal{\textbf{Web poslužitelj} je ključna komponenta u obradi korisničkih zahtjeva.
	Dakle, to je \textbf{središnji dio} aplikacije koji omogućava komunikaciju korisnika s aplikacijom.
	U središnjem sloju se odvijaju procesi koji su zaslužni za komunikaciju s bazom podataka ukoliko je to potrebno, a to se odvija
	prema MVC (Model - View - Controller) arhitekturnom obrascu. Model definira podatkovnu strukturu entiteta baze podataka,
	View (pogled) upravlja izgledom i prikazom sadržaja korisniku u web pregledniku, te Controller (kontroler) upravlja i reagira
	na korisničke događaje.
	Odnos komponenti predstavljen je slikom \ref{fig:arhitekturasustava2}.}

\begin{figure}[H]
	\includegraphics[width=\textwidth]{slike/arhitekturaSustava2.png} %veličina slike u odnosu na originalnu datoteku i pozicija slike
	\centering
	\caption{Arhitektura sustava backend}
	\label{fig:arhitekturasustava2}
\end{figure}

\textnormal{Za izradu ovakve web aplikacije kako bi se pokrile navedene specifične funkcionalnosti koristit će se Spring framework za Javu, te React za prikaz na web pregledniku u radnom okruženju IntelliJIDEA.
	Specifično za Spring framework, koristit će se navedeni tip arhitekture kao što je naveden slikom \ref{fig:arhitekturasustava2}, jer će se putem metoda JPARepository moći upravljati zahtjevima korisnika koji se vežu za upite u bazi podataka.
	Ukratko, obavljaju se operacije nad \textbf{PostgreSQL bazom podataka}.}

\textnormal{\textbf{Kontroler} upravlja korisničkim zahtjevima i prosljeđuje ih dalje prema \textbf{Service} gdje se obavlja logika nad upitima i zahtjevima. Service komunicira s \textbf{bazom podataka} preko JPARepository.}

\textnormal{UI će biti povezan sa ostalim dijelovima web aplikacije putem \textbf{REST servisa}, odnosno REST servis preko HTTP zahtjeva omogućuje pristup funkcijama za upravljanje bazom podataka. Nad bazom podataka se mogu vršiti  (CRUD - Create, Read, Update, Delete) operacije. Takva vrsta komunikacije je predstavljena slikom \ref{fig:arhitekturasustava3}.
	UI je izveden uz pomoć \textbf{React} framework-a koji se koristi komponentama za organizaciju prikaza.}

\begin{figure}[H]
	\includegraphics[width=\textwidth]{slike/arhitekturaSustava3.png} %veličina slike u odnosu na originalnu datoteku i pozicija slike
	\centering
	\caption{Arhitektura sustava backend i frontend}
	\label{fig:arhitekturasustava3}
\end{figure}

\eject





\section{Baza podataka}


\textnormal{Kao rješenje za web aplikaciju koju izrađujemo, odlučili smo se na relacijski model baze podataka jer nam ona omogućava precizno oblikovanje i modeliranje elemenata iz stvarnog svijeta. Međusobni odnos tablica se zasniva na relaciji dok se svaka tablica sastoji od naziva entiteta te njegovih pripadajućih atributa koji opisuju dani entitet. Ovakav tip baze podataka u ovom slučaju nam omogućava brzo spremanje, dohvat i izmjenu (uređivanje ili brisanje) podataka koji cirkulišu u web aplikaciji. Baza podataka koja je spremna za izradu ove web aplikacije se sastoji od sljedećih elemenata/tablica:}

\begin{packed_item}

	\item Korisnik
	\item Recept
	\item Kategorija
	\item ReceptKategorije
	\item ReceptSastojci
	\item Sastojak
	\item KomentariRecept
	\item Komentar
	\item Pratioci
	\item OznačavanjeRecepata
	\item SpremljenRecept
	\item VrsteKuhinjeRecepata
	\item VrstaKuhinja
	\item Obavijesti

\end{packed_item}

\eject

\subsection{Opis tablica}


\textnormal{\textbf{Korisnik}		Entitet Korisnik služi za evidenciju podataka o korisnicima koji koriste web aplikaciju. Atributi koji ga opisuju su korisnickoIme, lozinkaKorisnik, imeKorisnik, prezimeKorisnik, brojTelefona, emailKorisnik, razinaOvlasti i Dostupan što označava kada je dostupan za komunikaciju s drugim korisnicima unutar web aplikacije. Entitet Korisnik je u \textit{One-To-Many} vezi s entitetom Recept. Entitet Korisnik je i u \textit{Many-To-Many} vezi s entitetom OmiljeniAutor koji bilježi korisnike koje ovaj korisnik prati. Također, ovaj entitet je u \textit{Many-To-Many} vezi s entitetima SpremljenRecept, OznačenRecept, a u \textit{One-To-Many} vezi s entitetima Objava i Komentar.}


\begin{longtblr}[
	label=none,
	entry=none
	]{
	width = \textwidth,
	colspec={|X[8,l]|X[6, l]|X[20, l]|},
	rowhead = 1,
	} %definicija širine tablice, širine stupaca, poravnanje i broja redaka naslova tablice
	\hline \SetCell[c=3]{c}{\textbf{Korisnik}}                                     \\ \hline[3pt]
	\SetCell{LightGreen}IDKorisnik & INT     & jedinstveni identifikator korisnika \\ \hline
	KorisnickoIme                  & VARCHAR & korisničko ime                      \\ \hline
	LozinkaKorisnik                & VARCHAR & spremljena hash lozinka za pristup  \\ \hline
	ImeKorisnik                    & VARCHAR & ime korisnika                       \\ \hline
	PrezimeKorisnik                & VARCHAR & prezime korisnika                   \\ \hline
	BrojTelefona                   & VARCHAR & broj telefona korisnika             \\ \hline
	EmailKorisnik                  & VARCHAR & e-mail adresa korisnika             \\ \hline
	RazinaOvlasti                  & VARCHAR & razina ovlasti u aplikaciji         \\ \hline
	DostupanOd                     & TIME    & vrijeme dostupnosti                 \\ \hline
	DostupanDo                     & TIME    & vrijeme dostupnosti                 \\ \hline
\end{longtblr}

\eject

\textnormal{\textbf{Komentar}		Entitet Komentar služi za evidenciju komentara koji korisnici međusobno dijele u web aplikaciji. Opisuju ga atributi IDKomentar, IDKorisnik, SadrzajKomentar i DatumKomentar. Ovaj entitet je  u \textit{Many-To-One} vezi s entitetom Korisnik.}

\begin{longtblr}[
	label=none,
	entry=none
	]{
	width = \textwidth,
	colspec={|X[8,l]|X[6, l]|X[20, l]|},
	rowhead = 1,
	} %definicija širine tablice, širine stupaca, poravnanje i broja redaka naslova tablice
	\hline \SetCell[c=3]{c}{\textbf{Komentar}}                                     \\ \hline[3pt]
	\SetCell{LightGreen}IDKomentar & INT     & jedinstveni identifikator komentara \\ \hline
	\SetCell{LightBlue}IDKorisnik  & INT     & jedinstveni identifikator korisnika \\ \hline
	OpisKomentar                   & VARCHAR & sadržaj komentara                   \\ \hline
	DatumKomentar                  & DATE    & datum komentara                     \\ \hline
\end{longtblr}

\textnormal{\textbf{KomentariRecept}		Entitet KomentariRecept služi za evidenciju komentara. Ovaj entitet je  u \textit{Many-To-Many} vezi s entitetom Korisnik.}

\begin{longtblr}[
	label=none,
	entry=none
	]{
	width = \textwidth,
	colspec={|X[8,l]|X[6, l]|X[20, l]|},
	rowhead = 1,
	} %definicija širine tablice, širine stupaca, poravnanje i broja redaka naslova tablice
	\hline \SetCell[c=3]{c}{\textbf{KomentariRecept}}                          \\ \hline[3pt]
	\SetCell{LightGreen}ID         & INT & jedinstveni identifikator           \\ \hline
	\SetCell{LightGreen}IDKomentar & INT & jedinstveni identifikator komentara \\ \hline
	\SetCell{LightBlue}IDRecept    & INT & jedinstveni identifikator recepta   \\ \hline
\end{longtblr}


\vspace{\baselineskip}
\textnormal{\textbf{Pratioci}		Entitet Pratioci služi za evidenciju korisnika koje jedan korisnik zaprati. Opisan je sljedećim atributima: IDKorisnik i IDAutor. U \textit{Many-To-Many} je vezi s entitetom Korisnik unutar web aplikacije.}

\begin{longtblr}[
	label=none,
	entry=none
	]{
	width = \textwidth,
	colspec={|X[6,l]|X[6, l]|X[20, l]|},
	rowhead = 1,
	} %definicija širine tablice, širine stupaca, poravnanje i broja redaka naslova tablice
	\hline \SetCell[c=3]{c}{\textbf{Pratioci}}                                 \\ \hline[3pt]
	\SetCell{LightGreen}ID        & INT & jedinstveni identifikator            \\ \hline
	\SetCell{LightBlue}IDKorisnik & INT & jedinstveni identifikator pratitelja \\ \hline
	\SetCell{LightBlue}IDAutor    & INT & ID korisnika koji je zapraćen        \\ \hline
\end{longtblr}

\vspace{\baselineskip}
\textnormal{\textbf{Recept}		Entitet Recept služi za evidenciju podataka o receptima koji cirkulišu i nastaju u web aplikaciji. Opisan je atributima: IDRecept, NazivRecept, IDKategorija, IDSastojak, IDVrstaKuhinja, PripremaRecept, VrijemeKuhanja, IDOznaka, SlikaRecept i IDVideoRecept. Entitet Recept je u \textit{Many-To-One} vezi s entitetom Korisnik, a u \textit{Many-To-Many} vezi s entitetom Sastojak, te u \textit{Many-To-Many} vezi s Kategorija, a \textit{One-To-One} s VrstaKuhinja. Također je u \textit{One-To-One} vezi s entitetom Objava.}

\begin{longtblr}[
	label=none,
	entry=none
	]{
	width = \textwidth,
	colspec={|X[8,l]|X[6, l]|X[20, l]|},
	rowhead = 1,
	} %definicija širine tablice, širine stupaca, poravnanje i broja redaka naslova tablice
	\hline \SetCell[c=3]{c}{\textbf{Recept}}                                           \\ \hline[3pt]
	\SetCell{LightGreen}IDRecept      & INT     & jedinstveni identifikator recepta    \\ \hline
	NazivRecept                       & VARCHAR & naziv recepta                        \\ \hline
	\SetCell{LightBlue}IDKategorija   & INT     & ID kategorije kojoj recept pripada   \\ \hline
	\SetCell{LightBlue}IDSastojak     & INT     & ID sastojka u receptu                \\ \hline
	\SetCell{LightBlue}IDVrstaKuhinja & INT     & ID vrste kuhinje recepta             \\ \hline
	PripremaRecept                    & VARCHAR & Opis pripreme recepta                \\ \hline
	VrijemeKuhanja                    & TIME    & Vrijeme potrebno za pripremu recepta \\ \hline
	Oznaka                            & INT     & Oznaka recepta                       \\ \hline
	SlikaRecept                       & VARCHAR & Slika recepta                        \\ \hline
	VideoRecept                       & VARCHAR & Video recepta                        \\ \hline
\end{longtblr}

\vspace{\baselineskip}
\textnormal{\textbf{OznačavanjeRecepata}		Entitet OznačavanjeRecepata služi za evidenciju označenih recepata od strane korisnika. Opisan je atributima: IDRecept i IDKorisnik. U \textit{Many-To-Many} vezi je s entitetom Korisnik.}

\begin{longtblr}[
	label=none,
	entry=none
	]{
	width = \textwidth,
	colspec={|X[6,l]|X[6, l]|X[20, l]|},
	rowhead = 1,
	} %definicija širine tablice, širine stupaca, poravnanje i broja redaka naslova tablice
	\hline \SetCell[c=3]{c}{\textbf{OznačavanjeRecepata}}                     \\ \hline[3pt]
	\SetCell{LightGreen}ID        & INT & jedinstveni identifikator           \\ \hline
	\SetCell{LightBlue}IDRecept   & INT & jedinstveni identifikator recepta   \\ \hline
	\SetCell{LightBlue}IDKorisnik & INT & ID korisnika koji je označio recept \\ \hline
\end{longtblr}

\vspace{\baselineskip}
\textnormal{\textbf{SpremljenRecept}		Entitet SpremljenRecept za evidenciju spremljenih recepata od strane korisnika. Opisan je atributima: IDRecept i IDKorisnik. U \textit{Many-To-Many} vezi je s entitetom Korisnik.}

\begin{longtblr}[
	label=none,
	entry=none
	]{
	width = \textwidth,
	colspec={|X[6,l]|X[6, l]|X[20, l]|},
	rowhead = 1,
	} %definicija širine tablice, širine stupaca, poravnanje i broja redaka naslova tablice
	\hline \SetCell[c=3]{c}{\textbf{SpremljenRecept}}                         \\ \hline[3pt]
	\SetCell{LightGreen}ID        & INT & jedinstveni identifikator           \\ \hline
	\SetCell{LightBlue}IDRecept   & INT & jedinstveni identifikator recepta   \\ \hline
	\SetCell{LightBlue}IDKorisnik & INT & ID korisnika koji je spremio recept \\ \hline
\end{longtblr}

\vspace{\baselineskip}
\textnormal{\textbf{Kategorija}		Entitet Kategorija služi za evidenciju kategorija. Opisan je atributima: IDKategorija, NazivKategorija.}

\begin{longtblr}[
	label=none,
	entry=none
	]{
	width = \textwidth,
	colspec={|X[8,l]|X[6, l]|X[20, l]|},
	rowhead = 1,
	} %definicija širine tablice, širine stupaca, poravnanje i broja redaka naslova tablice
	\hline \SetCell[c=3]{c}{\textbf{Kategorija}}                                      \\ \hline[3pt]
	\SetCell{LightGreen}IDKategorija & INT     & jedinstveni identifikator kateogrije \\ \hline
	NazivKategorija                  & VARCHAR & naziv kategorije                     \\ \hline
\end{longtblr}

\vspace{\baselineskip}
\textnormal{\textbf{ReceptKategorije}		Entitet ReceptKategorije služi za evidenciju kategorija i recepata pod tim kategorijama. Opisan je atributima: IDKategorija, NazivKategorija. U \textit{Many-To-Many} vezi je s entitetom Recept.}

\begin{longtblr}[
	label=none,
	entry=none
	]{
	width = \textwidth,
	colspec={|X[8,l]|X[6, l]|X[20, l]|},
	rowhead = 1,
	} %definicija širine tablice, širine stupaca, poravnanje i broja redaka naslova tablice
	\hline \SetCell[c=3]{c}{\textbf{ReceptKategorije}}                           \\ \hline[3pt]
	\SetCell{LightGreen}ID          & INT & jedinstveni identifikator            \\ \hline
	\SetCell{LightBlue}IDRecept     & INT & jedinstveni identifikator kateogrije \\ \hline
	\SetCell{LightBlue}IDKategorija & INT & jedinstveni identifikator recepta    \\ \hline
\end{longtblr}

\vspace{\baselineskip}
\textnormal{\textbf{VrstaKuhinja}		Entitet VrstaKuhinja služi za evidenciju različitih vrsta kuhinja. Opisan je atributima: IDVrstaKuhinja, NazivVrstaKuhinja. U \textit{One-To-One} vezi je s entitetom Recept.}

\begin{longtblr}[
	label=none,
	entry=none
	]{
	width = \textwidth,
	colspec={|X[9,l]|X[6, l]|X[20, l]|},
	rowhead = 1,
	} %definicija širine tablice, širine stupaca, poravnanje i broja redaka naslova tablice
	\hline \SetCell[c=3]{c}{\textbf{VrstaKuhinja}}                                         \\ \hline[3pt]
	\SetCell{LightGreen}IDVrstaKuhinje & INT     & jedinstveni identifikator vrste kuhinje \\ \hline
	NazivVrstaKuhinja                  & VARCHAR & naziv vrste kuhinje                     \\ \hline
\end{longtblr}

\vspace{\baselineskip}
\textnormal{\textbf{VrsteKuhinjeRecepata}		Entitet VrsteKuhinjeRecepata služi za evidenciju različitih vrsta kuhinja recepta ukoliko je to prikladno za recept (nije određeno kojoj pripada).}

\begin{longtblr}[
	label=none,
	entry=none
	]{
	width = \textwidth,
	colspec={|X[9,l]|X[6, l]|X[20, l]|},
	rowhead = 1,
	} %definicija širine tablice, širine stupaca, poravnanje i broja redaka naslova tablice
	\hline \SetCell[c=3]{c}{\textbf{VrsteKuhinjeRecepata}}                                    \\ \hline[3pt]
	\SetCell{LightGreen}ID            & INT & jedinstveni identifikator               \\ \hline
	\SetCell{LightBlue}IDVrstaKuhinje & INT & jedinstveni identifikator vrste kuhinje \\ \hline
	\SetCell{LightBlue}IDRecept       & INT & jedinstveni identifikator recepta       \\ \hline
\end{longtblr}

\vspace{\baselineskip}
\textnormal{\textbf{Sastojak}		Entitet Sastojak služi za evidenciju različitih sastojaka. Opisan je atributima: IDSastojak, NazivSastojak.}

\begin{longtblr}[
	label=none,
	entry=none
	]{
	width = \textwidth,
	colspec={|X[6,l]|X[6, l]|X[20, l]|},
	rowhead = 1,
	} %definicija širine tablice, širine stupaca, poravnanje i broja redaka naslova tablice
	\hline \SetCell[c=3]{c}{\textbf{Sastojak}}                                    \\ \hline[3pt]
	\SetCell{LightGreen}IDSastojak & INT     & jedinstveni identifikator sastojka \\ \hline
	NazivSastojak                  & VARCHAR & naziv sastojka                     \\ \hline
\end{longtblr}

\vspace{\baselineskip}
\textnormal{\textbf{ReceptSastojci}		Entitet ReceptSastojci služi za evidenciju različitih sastojaka unutar recepta. Opisan je atributima: IDRecept, IDSastojak. U \textit{Many-To-Many} vezi je s entitetom Recept.}

\begin{longtblr}[
	label=none,
	entry=none
	]{
	width = \textwidth,
	colspec={|X[8,l]|X[6, l]|X[20, l]|},
	rowhead = 1,
	} %definicija širine tablice, širine stupaca, poravnanje i broja redaka naslova tablice
	\hline \SetCell[c=3]{c}{\textbf{ReceptSastojci}}                         \\ \hline[3pt]
	\SetCell{LightGreen}ID        & INT & jedinstveni identifikator          \\ \hline
	\SetCell{LightBlue}IDRecept   & INT & jedinstveni identifikator recepta  \\ \hline
	\SetCell{LightBlue}IDSastojak & INT & jedinstveni identifikator sastojka \\ \hline
\end{longtblr}

\vspace{\baselineskip}
\textnormal{\textbf{Obavijesti}		Entitet Obavijesti služi za slanje obavijesti korisnicima koji prate određene autore recepata. Opisan je atributima: IDObavijest, IDAutor, NazivObavijest, SadrzajObavijest, DatumObavijest i JeProcitano. U \textit{Many-To-One} vezi je s entitetom Korisnik.}

\begin{longtblr}[
	label=none,
	entry=none
	]{
	width = \textwidth,
	colspec={|X[8,l]|X[6, l]|X[20, l]|},
	rowhead = 1,
	} %definicija širine tablice, širine stupaca, poravnanje i broja redaka naslova tablice
	\hline \SetCell[c=3]{c}{\textbf{Obavijesti}}                                    \\ \hline[3pt]
	\SetCell{LightGreen}IDObavijest & INT     & jed. identifikator obavijesti       \\ \hline
	\SetCell{LightBlue}IDKorisnik   & INT     & jedinstveni identifikator korisnika \\ \hline
	NazivObavijest                  & VARCHAR & naziv obavijesti                    \\ \hline
	SadrzajObavijest                & VARCHAR & sadržaj obavijesti                  \\ \hline
	DatumObavijest                  & DATE    & datum obavijesti                    \\ \hline
	JeProcitano                     & TINYINT & oznaka pročitane obavijesti         \\ \hline
\end{longtblr}

\subsection{Dijagram baze podataka}
%unos slike
\begin{figure}[H]
	\includegraphics[width=\textwidth]{slike/CookBooked-dbp.png} %veličina slike u odnosu na originalnu datoteku i pozicija slike
	\centering
	\caption{Dijagram relacijske baze podataka za web aplikaciju}
	\label{fig:dijagrambp}
\end{figure}
\eject


\section{Dijagram razreda}

\begin{figure}[H]
	\includegraphics[width=\textwidth]{slike/DijagramRazreda.png} %veličina slike u odnosu na originalnu datoteku i pozicija slike
	\centering
	\caption{Dijagram razreda}
	\label{fig:dijagramraz}
\end{figure}

\begin{figure}[H]
	\includegraphics[width=\textwidth]{slike/ControllerRazredi.png} %veličina slike u odnosu na originalnu datoteku i pozicija slike
	\centering
	\caption{Dijagram razreda kontrolera}
	\label{fig:dijagramkontr}
\end{figure}

\begin{figure}[H]
	\includegraphics[width=\textwidth]{slike/DTO.png} %veličina slike u odnosu na originalnu datoteku i pozicija slike
	\centering
	\caption{DTO}
	\label{fig:dijagramdto}
\end{figure}

\textbf{\textit{dio 2. revizije}}\\

\textit{Prilikom druge predaje projekta dijagram razreda i opisi moraju odgovarati stvarnom stanju implementacije}



\eject

\section{Dijagram stanja}


	\noindent
	\textbf{Dijagram stanja - Dodavanje recepta}\newline
		{Registrirani korisnik šalje zahtjev za dodavanje recepta. Recept mora sadržavati ime, sastojke i korake pripreme. Dodatno, moguće je dodati video ili slike pripreme. Recept se sprema u bazu podataka.}

	\begin{figure}[H]
		\includegraphics[scale= 0.4]{slike/DijagramStanja.png}
		\centering
		\caption{Dijagram stanja}
		\label{fig:Dijagram stanja}
	\end{figure} 
\eject

\section{Dijagram aktivnosti}

	\noindent
	\textbf{Dijagram aktivnosti - Pregled recepta}\newline
		{Korisnik šalje zahtjev za receptima po nekom kriteriju. Aplikacija šalje upit bazi i na temelju odgovora prikazuje recepte korisniku. Tada, korinsik može spremiti recept ako je registriran. Ako nije, preusmjeri ga se na registraciju.}

	\begin{figure}[H]
		\includegraphics[scale= 0.4]{slike/Dijagram aktivnosti - Pregled Recepata.png}
		\centering
		\caption{Dijagram aktivnosti}
		\label{fig:Dijagram aktivnosti}
	\end{figure} 
\eject

\section{Dijagram komponenti}


{Sustav se sastoji od \textit{frontend} i \textit{backend} dijelova, te baze podataka. \textit{Frontend} dio sastoji se od niza JSX datoteka koje određuju prikaz stranice (HomePage, Login) ili dijela stranice (Sidebar). React-view komponenta komunicira sa komponentom Router i \textit{backendom} te ovisno o korisnikovim akcijama osvježava prikaz i dohvaća nove podatke ili datoteke. Komponenta Router na upit sa URL-om odlučuje koja datoteka će se poslužiti na sučelje. JSX datoteke ovise o React i Redux bibliotekama. \textit{Frontend} uspostavlja komunikaciju sa backendom koristeći REST API. Zahtjev se proslijeđuje Controllers komponenti koja koristi metode izložene u sučelju Services komponente. Komponenta Services ostvaruje komunikaciju sa komponentom Repositories koja je zadužena za dohvat tablica iz baze podataka.}

	\begin{figure}[H]
		\includegraphics[scale= 0.45]{slike/Component Diagram0.png}
		\centering
		\caption{Dijagram komponenti}
		\label{fig:Dijagram komponenti}
	\end{figure} 