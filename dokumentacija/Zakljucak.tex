\chapter{Zaključak i budući rad}
		 
		 \noindent Kroz trajanje cijelog ovog projekta, nailazili smo na brojne prepreke. Prva od njih je naravno bila usklađivanje korištenja gita sa svim članovima tima. Tu se događaju gubitci i korištenje reverta i slično, dok se nismo bolje upoznali sa opcijom merge. Merge kod izrade tehničke dokumentacije bih rekao da je bio izazovan zbog brojnih konflikata kod slika i pdf dokumenata i neiskustva kod bavljenja s istim. \\
		 
		 \noindent Prelaskom na izradu aplikacije prvi problem na koji smo naišli bilo je uspostava backenda i spajanje Spring Boota na bazu podataka. Kada smo to riješili došlo je vrijeme za deploy aplikacije na web pomoću web aplikacije Render. Tu su krenuli brojni problemi najviše greška CORS koja nije dozvoljavala na frontendu fetchanje backenda koji nije u istoj domeni.\\
		 
		 \noindent Taj problem rješili smo u dva dijela. Prvo, kada je proradilo, no i dalje je postojao određeni time lag te tada uz drugi puta uz pomoć korištenja axiosa i reduxa dobili smo efektivno i funkcionalno rješenje. No ta promjena uzrokovala je i promjenu sustava dohvaćanja podataka, što je iziskivalo upoznavanje sa našim novim sistemom dohvaćanja podataka sa backenda.\\
		 
		 \noindent Kroz proces izrade cijele aplikacije, upoznali smo se s brojnim problemima sa kojima ćemo se susretati na kasnijim radovima, od git-a, do zahtjevnosti pisanja dokumentacije i neke od osnovnih stvari koje se često zanemaruju, a to je koliko je značajno i olakšava posao rad u timu. Kvalitetan rad u timu može uvelike ubrzati izradu aplikacije te olakšati teret što uvelike daje svakom članu time veću motivaciju za rad.\\
		 
		 \noindent Naravno, stekli smo i znanja tehnologija React i Spring Boot, te kako izgleda upload aplikacije na web. Osim toga po prvi puta smo se susreli sa alatom za izradu pdf dokumenta, latex, što je sam po sebi zanimljiv i vrlo precizan alat.\\
		 
		 \noindent Sve u svemu, smatramo kako smo stekli veliko iskustvo, koliko programsko riješenje za web aplikaciju iziskuje vremena da se napravi u cjelosti te koliko različitih ljudi može stajati iza jedne uspješne aplikacije sa velikim brojem korisnika.\\
		
		 \noindent U aplikaciji nije implementiran video kao ni videopoziv između korisnika.
		
		\eject 