\chapter{Specifikacija programske potpore}

\section{Funkcionalni zahtjevi}

\textbf{\textit{dio 1. revizije}}\\

\textit{Navesti \textbf{dionike} koji imaju \textbf{interes u ovom sustavu} ili  \textbf{su nositelji odgovornosti}. To su prije svega korisnici, ali i administratori sustava, naručitelji, razvojni tim.}\\

\textit{Navesti \textbf{aktore} koji izravno \textbf{koriste} ili \textbf{komuniciraju sa sustavom}. Oni mogu imati inicijatorsku ulogu, tj. započinju određene procese u sustavu ili samo sudioničku ulogu, tj. obavljaju određeni posao. Za svakog aktora navesti funkcionalne zahtjeve koji se na njega odnose.}\\


\noindent \textbf{Dionici:}

\begin{packed_enum}

	\item Dionik 1
	\item Dionik 2
	\item ...

\end{packed_enum}

\noindent \textbf{Aktori i njihovi funkcionalni zahtjevi:}


\begin{packed_enum}
	\item  \underbar{Aktor 1 (inicijator) može:}

	\begin{packed_enum}

		\item funkcionalnost 1
		\item funkcionalnost 2
		\begin{packed_enum}

			\item  podfunkcionalnost 1
			\item  podfunkcionalnost 2

		\end{packed_enum}
		\item  funkcionalnost 3

	\end{packed_enum}

	\item  \underbar{Aktor 2 (sudionik) može:}

	\begin{packed_enum}

		\item funkcionalnost 1
		\item funkcionalnost 2

	\end{packed_enum}
\end{packed_enum}

\eject



\subsection{Obrasci uporabe}


\noindent \underbar{\textbf{UC1 - Pregled recepata}}
\begin{packed_item}

	\item \textbf{Glavni sudionik: Neregistiran korisnik, Registriran korisnik}
	\item  \textbf{Cilj: Pregled recepata temeljem kategorija}
	\item  \textbf{Sudionici: Baza podataka}
	\item  \textbf{Preduvjet: -}
	\item  \textbf{Opis osnovnog tijeka:}

	\item[] \begin{packed_enum}

		\item Korisnik otvori platformu
		\item Ponuđene su mu kategorije, vrste kuhinje, specifični sastojci
		\item Korisnik odabire kategoriju i prikazuju mu se recepti
	\end{packed_enum}

	\item  \textbf{Opis mogućih odstupanja:}

	\item[] \begin{packed_item}

		\item[2.a]
		\item[] \begin{packed_enum}

			\item
			\item

		\end{packed_enum}
		\item[2.b]
		\item[3.a]

	\end{packed_item}
\end{packed_item}


\noindent \underbar{\textbf{UC2 - Registracija}}
\begin{packed_item}

	\item \textbf{Glavni sudionik: Neregistrirani korisnik}
	\item  \textbf{Cilj: Stvaranje računa za pristup ostalim značajkama sustava}
	\item  \textbf{Sudionici: Baza podtaka}
	\item  \textbf{Preduvjet:-}
	\item  \textbf{Opis osnovnog tijeka:}

	\item[] \begin{packed_enum}

		\item Korisnik odabire opciju za registraciju
		\item Korisnik unosi potrebne korisničke podatke
		\item Korisnik prima obavijest o uspješnoj registraciji
	\end{packed_enum}

	\item  \textbf{Opis mogućih odstupanja:}

	\item[] \begin{packed_item}

		\item[2.a] Odabir već zauzetog korisničkog imena i/ili e-maila, unos korisničkog podatka u nedozvoljenom formatu ili pružanje neispravnoga e-maila
		\item[] \begin{packed_enum}

			\item  Sustav obavještava korisnika o neuspjelom upisu i vraća ga na stranicu za registraciju
			\item Korisnik mijenja potrebne podatke te završava unos ili odustaje od registracije

		\end{packed_enum}

	\end{packed_item}
\end{packed_item}


\noindent \underbar{\textbf{UC3 - Prijava u sustav}}
\begin{packed_item}

	\item \textbf{Glavni sudionik: }
	\item  \textbf{Cilj:}
	\item  \textbf{Sudionici:}
	\item  \textbf{Preduvjet:}
	\item  \textbf{Opis osnovnog tijeka:}

	\item[] \begin{packed_enum}

		\item
		\item
		\item
		\item
		\item
	\end{packed_enum}

	\item  \textbf{Opis mogućih odstupanja:}

	\item[] \begin{packed_item}

		\item[2.a]
		\item[] \begin{packed_enum}

			\item
			\item

		\end{packed_enum}
		\item[2.b]
		\item[3.a]

	\end{packed_item}
\end{packed_item}

\noindent \underbar{\textbf{UC4 - Objava recepata}}
\begin{packed_item}

	\item \textbf{Glavni sudionik: Registrirani korisnik }
	\item  \textbf{Cilj: Objava svog recepta na stranici}
	\item  \textbf{Sudionici:}
	\item  \textbf{Preduvjet: Registracija i prijava korisnika}
	\item  \textbf{Opis osnovnog tijeka:}

	\item[] \begin{packed_enum}

		\item
		\item
		\item
		\item
		\item
	\end{packed_enum}

	\item  \textbf{Opis mogućih odstupanja:}

	\item[] \begin{packed_item}

		\item[2.a]
		\item[] \begin{packed_enum}

			\item
			\item

		\end{packed_enum}
		\item[2.b]
		\item[3.a]

	\end{packed_item}
\end{packed_item}

\noindent \underbar{\textbf{UC5 - Razmjena poruka}}
\begin{packed_item}

	\item \textbf{Glavni sudionik: }
	\item  \textbf{Cilj:}
	\item  \textbf{Sudionici:}
	\item  \textbf{Preduvjet:}
	\item  \textbf{Opis osnovnog tijeka:}

	\item[] \begin{packed_enum}

		\item
		\item
		\item
		\item
		\item
	\end{packed_enum}

	\item  \textbf{Opis mogućih odstupanja:}

	\item[] \begin{packed_item}

		\item[2.a]
		\item[] \begin{packed_enum}

			\item
			\item

		\end{packed_enum}
		\item[2.b]
		\item[3.a]

	\end{packed_item}
\end{packed_item}

\noindent \underbar{\textbf{UC6 - Čavrljanje }}
\begin{packed_item}

	\item \textbf{Glavni sudionik: }
	\item  \textbf{Cilj:}
	\item  \textbf{Sudionici:}
	\item  \textbf{Preduvjet:}
	\item  \textbf{Opis osnovnog tijeka:}

	\item[] \begin{packed_enum}

		\item
		\item
		\item
		\item
		\item
	\end{packed_enum}

	\item  \textbf{Opis mogućih odstupanja:}

	\item[] \begin{packed_item}

		\item[2.a]
		\item[] \begin{packed_enum}

			\item
			\item

		\end{packed_enum}
		\item[2.b]
		\item[3.a]

	\end{packed_item}
\end{packed_item}

\noindent \underbar{\textbf{UC7 - Videopoziv}}
\begin{packed_item}

	\item \textbf{Glavni sudionik: }
	\item  \textbf{Cilj:}
	\item  \textbf{Sudionici:}
	\item  \textbf{Preduvjet:}
	\item  \textbf{Opis osnovnog tijeka:}

	\item[] \begin{packed_enum}

		\item
		\item
		\item
		\item
		\item
	\end{packed_enum}

	\item  \textbf{Opis mogućih odstupanja:}

	\item[] \begin{packed_item}

		\item[2.a]
		\item[] \begin{packed_enum}

			\item
			\item

		\end{packed_enum}
		\item[2.b]
		\item[3.a]

	\end{packed_item}
\end{packed_item}


\subsubsection{Dijagrami obrazaca uporabe}

\textit{Prikazati odnos aktora i obrazaca uporabe odgovarajućim UML dijagramom. Nije nužno nacrtati sve na jednom dijagramu. Modelirati po razinama apstrakcije i skupovima srodnih funkcionalnosti.}
\eject

\subsection{Sekvencijski dijagrami}

\noindent
\textbf{Obrazac uporabe UC1-Pregled recepata}\newline
{Korisnik šalje zahtjev za prikaz recepata po kategorijama, sastojcima i/ili kuhinjama kojim pripadaju. Poslužitelj dohvaća recepete koji zadovoljavaju uvjete i prikazuje ih korisniku. Korisnik sada može spremiti recepte, ako je prijavljen to se provodi, ako nije preusmjeri ga se na stranicu za prijavu.}


\begin{figure}[H]
	\includegraphics[scale= 0.4]{slike/sekvencijski_dijagramUC1.png}
	\centering
	\caption{Sekvencijski dijagram za UC1}
	\label{fig:Sekvencijski dijagram za UC1}
\end{figure}
\eject

\noindent
\textbf{Obrazac uporabe UC3-Registracija}\newline
{Korisnik se registrira s korisničkim imenom i lozinkom. Ako takav korisnik već postoji, korisniku se ispisuje greška. Inače, korisnik se uspješno registrirao i to se bilježi u bazi podataka.}


\begin{figure}[H]
	\includegraphics[scale= 0.6]{slike/sekvencijski_dijagramUC3.png}
	\centering
	\caption{Sekvencijski dijagram za UC3}
	\label{fig:Sekvencijski dijagram za UC3}
\end{figure}

\eject

\noindent
\textbf{Obrazac uporabe UC4-Prijava Korisnika}\newline
{Korisnik pokuša objaviti novi recept. Ako nije prijavljen, preusmjeri ga se na prijavu. Inače, recept se dodaje u bazu podataka i o tome se obavijesti korisnik.}


\begin{figure}[H]
	\includegraphics[scale= 0.6]{slike/sekvencijski_dijagramUC4.png}
	\centering
	\caption{Sekvencijski dijagram za UC4}
	\label{fig:Sekvencijski dijagram za UC4}
\end{figure}

\eject

\noindent
\textbf{Obrazac uporabe UC8-Videopoziv}\newline
{Korisnik zatraži video poziv s drugim korisnikom. Ako drugi korisnik ne postoji, poziv se ne uspostavlja i o tome se obavijesti prvog korisnika. Inače se šalje zahtjev za video pozivom drugom korisniku, koji ga može odbiti ili prihvatiti. Ako odbije poziv se ne uspostavlja i o tome se obavijesti prvi korsnik. Ako prihvati uspostavi se video poziv između dva korisnika.}


\begin{figure}[H]
	\includegraphics[scale= 0.5]{slike/sekvencijski_dijagramUC7.png}
	\centering
	\caption{Sekvencijski dijagram za UC7}
	\label{fig:Sekvencijski dijagram za UC7}
\end{figure}

\eject

\section{Ostali zahtjevi}

\textit{Nefunkcionalne zahtjeve koji će se navesti u nastavku teksta pojašnjavaju dodatne zahtjeve koje web aplikacija treba koristiti ili već koristi.}

\begin{packed_item}

	\item Sustav treba biti implementiran kao web aplikacija koristeći objektno-orijentirane paradigme i norme kako bi se omogućilo ponovno korištenje dijelova koda/modula.
	\item Nepravilno korištenje web aplikacije ne smije rezultirati padom sustava, odnosno potrebno je usmjeriti korisnika na pravilno korištenje pop-up dijalozima.
	\item U sustavu treba biti omogućen rad i korištenje web aplikacije od strane više korisnika (cca. 50).
	\item Sustav mora podržati dijakritičke znakove hrvatskog jezika, dakle treba podržati hrvatsku abecedu. Omogućit će se i promjena jezika s hrvatskog na engleski jezik.
	\item Hrvatski jezik je zadani jezik unutar web aplikacije.
	\item Sustav mora imati intuitivno sučelje koje neće stvarati nedoumice kod korisnika odnosno web-aplikacija treba biti jednostavna za korištenje.
	\item Konekcija s bazom podataka mora imati brz odziv. Bilo kakav pokušaj neovlaštenog pristupa informacijama u bazi podataka potrebno je spriječiti i istu adekvatno zašititi.
	\item Web aplikacija koristi proces kripitiranja lozinki za prijavu korisnika koji se tako u hash-ovima spremaju u bazu podataka.
	\item Sustav mora omogućiti korištenje određenih funkcionalnosti samo prijavljenim/registriranim korisnicima.
	\item Učitavanje početne stranice web aplikacije ne smije trajati duže od 5 sekundi.
	\item Pristup sustavu i razmjena podataka se vrši HTTPS protokolom.
	\item Pri razvoju web aplikacije koristi se React Native i Spring framework.

\end{packed_item}



